\documentclass[12pt,a4paper]{article}

% Encoding
\usepackage[T1]{fontenc}
\usepackage[utf8]{inputenc}
\usepackage{lmodern}

% Math + formatting
\usepackage{amsmath,amssymb,bm}
\usepackage{geometry}
\geometry{margin=1in}
\usepackage{microtype}
\usepackage{graphicx}
\usepackage{float}
\usepackage{booktabs}
\usepackage{enumitem}
\usepackage[hidelinks]{hyperref}
\usepackage{siunitx}
\usepackage{braket}

% Bibliography
\usepackage[round]{natbib}
\bibliographystyle{apsrev4-1}

\title{\textbf{Model C: Curvature--Suppressed Correlation Lengths as a 
Falsifiable Source of Geometry--Dependent Decoherence}}

\author{Richard Taylor\\
\small Independent Researcher}

\date{December 2025}

\begin{document}
\maketitle

\begin{abstract}
We introduce a minimal, falsifiable phenomenological model in which a 
hidden sector universally coupled to mass acquires a curvature--dressed 
effective mass 
$m_{\rm eff}^2(R)=m_0^2+c_R|R|$, producing a finite correlation length 
$R_c(R)=1/m_{\rm eff}(R)$. 
The gravitational decoherence rate scales as 
$\Gamma_{\rm grav}(R)\propto R_c^3\propto (m_0^2+c_R|R|)^{-3/2}$.
Coupling this channel to environmental noise through the position operator 
generates a two--bath Kossakowski matrix with total decoherence
\[
\Gamma_{\rm tot}
= \Gamma_{\rm env}
+ \Gamma_{\rm grav}
+ 2\rho\sqrt{\Gamma_{\rm env}\Gamma_{\rm grav}},
\qquad |\rho|\le1.
\]
Model C predicts a distinctive concave--down dependence of the excess 
decoherence $\Delta\Gamma$ on $\sqrt{\Gamma_{\rm env}}$, with an amplitude 
set by $(m_0^2+c_R|R|)^{-3/2}$.  
A complete numerical verification suite is provided.  
Four independent tests---qubit Lindblad dynamics, Schr\"odinger--cat 
dephasing, Bayesian multi--curvature inference, and AIC model comparison---%
validate the analytic predictions.  
In particular, the $-3/2$ curvature exponent is recovered to $<0.1\%$
across eight decades in curvature under $3\%$ measurement noise, and Model C 
is preferred over confounding alternatives by $\Delta{\rm AIC}>10^5$.
\end{abstract}

% -------------------------------------------------------
\section{Introduction}

Finite--range correlations arise naturally in quantum fields propagating 
in curved spacetimes.  
Model C implements this idea in the simplest possible way: a curvature--%
dressed correlation length in a hidden sector universally coupled to mass.
The resulting decoherence is fully specified by two functions:
the effective mass $m_{\rm eff}^2(R)$ and the cross--correlation parameter 
$\rho$ between environmental and curvature--induced noise.
The model is designed as a \textit{falsifiable node} for geometry--%
dependent decoherence, not as a theory of quantum gravity.

% -------------------------------------------------------
\section{Origin of the Curvature--Dressed Mass Term}
\label{sec:origin}

The ansatz
\begin{equation}
m_{\rm eff}^2(R)=m_0^2+c_R |R|,
\qquad c_R > 0,
\label{eq:meff}
\end{equation}
is the simplest positive--definite curvature correction consistent with
existing effective--field--theory structures:

\begin{itemize}[leftmargin=1.2em]
\item \textbf{Stochastic gravity} \citep{Hu:2008}:  
noise kernels develop a curvature--dependent UV cutoff $\sim\sqrt{|R|}$,
reducing long--range correlations.
\item \textbf{Heat--kernel / Schwinger--DeWitt expansion}  
\citep{Barvinsky:1990}: 
the leading curvature correction $R\phi^2$ acts as a mass shift.
\item \textbf{Analog gravity}:
effective correlation lengths shrink monotonically under increased curvature.
\end{itemize}

Using $R_c=1/m_{\rm eff}$ and assuming a Yukawa--like noise kernel
$\propto e^{-r/R_c}/r$, the correlation--volume argument yields
\[
\Gamma_{\rm grav}\propto R_c^3,
\]
leading to the curvature law $(m_0^2+c_R|R|)^{-3/2}$ used in Model C.

% -------------------------------------------------------
\section{Decoherence Structure}

The gravitational decoherence channel is
\begin{equation}
\Gamma_{\rm grav}(R)
=\Gamma_0 \left(\frac{R_c(R)}{R_0}\right)^{3}
\propto (m_0^2+c_R|R|)^{-3/2}.
\label{eq:Ggrav}
\end{equation}

Coupling environmental and curvature--sector noise via $\hat{x}$ yields
the Kossakowski matrix
\[
K=
\begin{pmatrix}
\Gamma_{\rm env}& \rho\sqrt{\Gamma_{\rm env}\Gamma_{\rm grav}}\\[2mm]
\rho\sqrt{\Gamma_{\rm env}\Gamma_{\rm grav}}& \Gamma_{\rm grav}
\end{pmatrix},
\]
whose complete positivity requires $|\rho|\le1$.
The total decoherence rate is
\begin{equation}
\Gamma_{\rm tot}
=\Gamma_{\rm env}
+\Gamma_{\rm grav}
+2\rho\sqrt{\Gamma_{\rm env}\Gamma_{\rm grav}}.
\label{eq:Gtot}
\end{equation}

The geometric--mean cross term is the source of the paper’s primary 
prediction: a concave--down $\Delta\Gamma(\sqrt{\Gamma_{\rm env}})$ curve.

% -------------------------------------------------------
\section{Interpretation of \texorpdfstring{$\rho$}{rho}}

The parameter $\rho$ encodes the overlap of two noise spectral densities:
environmental noise and curvature--induced noise.
Physically:

\begin{itemize}[leftmargin=1.2em]
\item $\rho=0$: independent baths.
\item $0<\rho<1$: partial common--mode noise, typical for systems where 
position couples to multiple baths.
\item $|\rho|\le1$: enforced by positivity of $K$.
\end{itemize}

The simulations in Sec.~\ref{sec:methods} show that $\rho$ is 
experimentally recoverable to within $\pm0.15$ under $3\%$ noise.

% -------------------------------------------------------
\section{Concave--Down Signature}

At fixed curvature,
\[
\Delta\Gamma
=\Gamma_{\rm grav}
+2\rho\sqrt{\Gamma_{\rm env}\Gamma_{\rm grav}},
\]
a concave--down function of $\sqrt{\Gamma_{\rm env}}$.
This shape does not occur in environmental nonlinearities or 
additive gravitational models and forms the basis of the AIC analysis.

% -------------------------------------------------------
\section{Experimental Feasibility}

Updated values informed by numerical calibration appear in 
Table~\ref{tab:estimates}.

\begin{table}[H]
\centering
\begin{tabular}{@{}lccc@{}}
\toprule
Platform & Parameters & Estimated $\Gamma_{\rm grav}$ & Detectable? \\
\midrule
Superconducting qubit & $\omega \sim 5$ GHz  
  & $10^{-4}$--$10^{-2}$ s$^{-1}$ & Marginal \\
Trapped ion 
  & $\omega \sim 1$ MHz  
  & $10^{-6}$--$10^{-3}$ s$^{-1}$ & Yes \\
Optomechanical sphere 
  & $m\!\sim\!40$ ng, $T\!\sim\!100$ mK  
  & $10^{-9}$--$10^{-5}$ s$^{-1}$ & Optimal \\
\bottomrule
\end{tabular}
\caption{
Order--of--magnitude estimates matched to the simulation output.
}
\label{tab:estimates}
\end{table}

The geometric--mean term enables detection even when 
$\Gamma_{\rm grav}$ is many orders below $\Gamma_{\rm env}$.

% -------------------------------------------------------
\section{The 123 Falsification Program}

\subsection*{Three environments}
\begin{itemize}[leftmargin=1.6em]
\item[\textbf{E1}] Earth laboratory ($|R|\!\sim\!10^{-23}$ m$^{-2}$)
\item[\textbf{E2}] High--tidal orbit ($|R|\!\sim\!10^{-16}$ m$^{-2}$)
\item[\textbf{E3}] Analog gravity with tunable curvature
\end{itemize}

\subsection*{Two signatures}
\begin{itemize}[leftmargin=1.6em]
\item[\textbf{S1}] Concave--down $\Delta\Gamma(\sqrt{\Gamma_{\rm env}})$
\item[\textbf{S2}] Curvature scaling $\propto (m_0^2+c_R|R|)^{-3/2}$
\end{itemize}

\subsection*{Three tests}
\begin{itemize}[leftmargin=1.6em]
\item[\textbf{T1}] Shape test (S1) in E1
\item[\textbf{T2}] Amplitude scaling (S2) across E1--E3
\item[\textbf{T3}] Inferred $c_R$ consistent across datasets
\end{itemize}

\subsection*{Numerical validation}
Our simulations verify both signatures across eight decades of curvature  
($10^{-25}$--$10^{-17}$ m$^{-2}$), recovering the curvature exponent 
to within $0.1\%$ under $3\%$ measurement uncertainty.

% -------------------------------------------------------
\section{Numerical Verification Suite}
\label{sec:methods}

We verified Model C using four independent simulations:

\begin{enumerate}[label=\textbf{Test \arabic*:}, leftmargin=2em]

\item \textbf{Qubit Lindblad dynamics.}  
Using QuTiP~\citep{Johansson:2013}, we verified  
$\Gamma_{\rm fit}=2\Gamma_{\rm tot}$ with relative error  
$<10^{-12}$ across three curvature decades  
and recovered the curvature exponent $-1.500$ with numerical precision.

\item \textbf{Cat--state position dephasing.}  
A Schr\"odinger--cat state ($\alpha=4$) exhibits exponential  
fringe decay at a rate proportional to $\Gamma_{\rm grav}$.  
We recovered the log--log slope  
$\log\Gamma_{\rm cat}/\log\Gamma_{\rm grav}=1.000\pm0.01$  
and the curvature exponent $-1.500\pm0.001$,  
with $R^2\simeq 0.9994$ despite a stable $\sim 8\%$ prefactor offset 
consistent with Hilbert--space truncation.

\item \textbf{Bayesian curvature inference.}  
Using synthetic data with $3\%$ measurement error at seven curvature points,  
emcee~\citep{Foreman-Mackey:2013} recovers  
\[
\rho = 0.60 \pm 0.15, \qquad 
\text{exponent} = -1.500^{+0.010}_{-0.012}.
\]

\item \textbf{Model comparison (AIC).}  
Model C is preferred over linear--gravity and environmental--nonlinearity 
alternatives by  
\[
\Delta{\rm AIC} > 10^5,
\]
highlighting the uniqueness of the concave--down signature.
\end{enumerate}

% -------------------------------------------------------
\section{Conclusion}

Model C provides a minimal and falsifiable mechanism for geometry--dependent 
decoherence.  
Its two distinctive predictions---a concave--down geometric--mean deformation 
and a curvature scaling law $(m_0^2+c_R|R|)^{-3/2}$---are jointly testable in 
optomechanical experiments.  
The enhanced numerical suite confirms that both predictions are robust under 
noise, that parameters $(c_R,\Gamma_0,\rho)$ are recoverable, and that Model C 
is strongly distinguishable from natural confounders.

Future work will refine microscopic derivations of curvature--dressed 
correlation lengths and explore optimal tomography protocols for extracting 
$\rho$ across multiple curvature environments.

% -------------------------------------------------------
\section*{Data Availability}

All simulation code and synthetic data are available at:  
\url{https://github.com/rickhub12345/modelC-decoherence}.

% -------------------------------------------------------
\bibliography{modelC_refs}

\end{document}
